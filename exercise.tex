\documentclass{article}
\begin{document}

\section{Motivation}
With recent events privacy and the safekeeping of information have become a
much debated topic. However, we have voluntarily been giving companies
hordes of information about ourselves for a long time, and in many
countries governments collect data on their citizens right from
the moment they are born.
All of it a long time before privacy became a 'hot topic'.

On one side of the argument stands the advocates of efficiency
and personalisation; knowing more about the preferences of a person
allows internet behaviour to be tailored to the user.
This makes use of the internet faster and more precise which i
favourable to most.
On the other hand privacy advocates warns against the loss of a
private sphere and a fundamental harm towards trust in the
society in general.

But when is it \textit{okay} to collect data, and when is it not?
What information about us are we willing to share, and to whom?

The question I would like to treat has to do with the boundary between 
\textit{harmful} and \textit{harmless} data; what types of data do we
consider as \textit{private}? And when are these data-types considered 
to be \textit{in the wrong hands}? 

Does the end user licenses agreement (EULA\footnote{A license description
where a service declares the rights of the customer, the ownership of
the data and other legal issues.}) solve all issues of sharing because
the terms are \textit{agreed upon}?

\section{Implication}
This question can either be adressed both by meticulous
classification and ethical discussions. I would like to scratch the 
surface of both methods by starting out with a crude categorisation 
of \textit{data} in a broad sense and then examine the different
ethical standings of what we might understand as the privacy of the
individual. Lastly I would investigate whether any demarcation
on types of data versus harmfullness of data can be made.

\section{Writing exercise}
The idea could be coupled with the concept or trust and authority. 
Example: When Nets was sold to an (American) company. Perhaps there are
degrees of trust, so the closer tothe individual the more trustworthy 
a company or 'authority' is. 

I define harmless data as information about a given individual that the 
individual considers as public.
Harmful data is information that the same individual considers as not 
relevant for the public and of possible harm for the person, if distributed.

This distinction leaves room for information that overlaps, i. e. that is
both public but harmful, or private but harmless.

The categorisation uses the above definitions to create a discrete 
distinction that can be measured. This division inevitably misses 
relevant details, but for the sake of this assignment I chose...

\end{document}
