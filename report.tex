\documentclass{article}

\usepackage{natbib}
\bibliographystyle{plainnat}
\usepackage{url}
\usepackage{nameref}

\begin{document}

\section{Introduction}
In an old Danish sketch about a raft, actor \texttt{A} tries to explain a theatrical scene to another actor, \texttt{B}. \texttt{A} states that a scene with a raft ends in a humorous climax. \texttt{B} then asks what a climax is and actor \texttt{A} defines it as an abstract peak of events. \texttt{B} nods and repeats that a climax, then, must be an abstract mountain. \texttt{B} asks what \textit{abstract} means and explains that it is something that is not there. \texttt{B} can now safely conclude that the scene contains a mountain that is not there. Needless to say, the conversation ends in complete confusion \citep{Raft:2014}.

The problem, of course, is that the two actors does not speak the same language, or, in terms of Kuhn, disciplinary matrix (see section \ref{Theory:Paradigms} on \nameref{Theory:Paradigms}, \cite{Holm:2014}).

In its most fundamental form the question of ownership of land, food and primary resources has existed for a
very very long time. Over the course of history, moral and cultural rules have evolved to address the
many confusion and strives over this matter. Faced with immaterialities such as
ideas and intellectual properties however, 'who owns what' becomes an increasingly hard question to answer. Ideas can be thought of by anyone, but the practical execution of them requires work from the physical domain. A domain where \texttt{you}

If intellect can be owned by everyone, and intellect can reason about ideas, how can ideas be owned? 

Copyright and software ownership can be considered as 

\section{Theory}


\subsection{Paradigms} \label{Theory:Paradigms}

\subsection{Society}

\section{Analysis}

\section{Discussion}

\section{Conclusion}
In this paper I have shown that...
By analysing and discussing the question from the perspective
of philosophy of science I have ...

\bibliography{references}

\end{document}
