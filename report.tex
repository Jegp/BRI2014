\documentclass{article}

\usepackage{natbib}
\bibliographystyle{plainnat}
\usepackage{url}
\usepackage{nameref}

\begin{document}

\section{Introduction}
In an old Danish sketch about a sailing raft on actor, actor \texttt{A}, tries to explain a theatrical scene to another actor, actor \texttt{B}. The sketch begins by \texttt{A} explaining how a scene with a raft ends in a humorous climax. \texttt{B} stops \texttt{A} in his storytelling to asks what a 'climax' means. Actor \texttt{A} explains (slightly annoyed) that a climax is as an abstract peak of events. \texttt{B} nods and repeats that the raft sketch contains an abstract mountain. \texttt{B} then asks what \textit{abstract} means and \texttt{A} (even more annoyed) explains that \textit{abstract} is something that is not there. The confusion is complete when \texttt{B} surely concludes that the raft scene contains a mountain that is not there. Needless to say, the conversation does not benefit either \texttt{A} or \texttt{B} \citep{Raft:2014}.

This type of '\textit{raft-talk}' is not uncommon when experts in information systems (IS) encounters people who does not have the same understandings of information technology. One example is a debate between a Danish politician and a Danish blogger about the recent surveillance scandals \citep{Bramsen:2014}. The interview has been heavily debated on social media and has recently led to some aggressive outburst from the IS field [ibid.]. 

It is not the first time the field of IS feels misunderstood % TODO: Find referencer
and, assuming the gradual introduction of IS in the general society continues, it will most certainly not be the last.
The question I would like to address in this essay is how the field of IS can be discussed by someone without expert knowledge:
\begin{quote}
How can the field of IS be approached by actors without expert knowledge within IS?
\end{quote}

To answer this question I have chosen two theoretical approaches: paradigms and computer ethics. 
By using Wernick et al. and Thomas Kuhn's theory on paradigms \citep{Wernick:2004, Holm:2014}, I will examine if, and if is how, the field of software engineering (SE) and IS can be understood as incommensurable paradigms.
By applying utilitarianism and Moore's text on 'computer ethics' \citep{Moor:1985}, I will discuss the above ideas and test if the 'old' non-digital society have any common grounds with the 'new' digital.
To test these approaches I will use the debate between the blogger and the politician mentioned above as a practical case in the analyses. Lastly I will discuss the two analyses and outline methods for further studies.

\section{Theory}
In this section I will include the theory about paradigms from Thomas Kuhn, and utilitarianism from Bentham and Mill. Before moving on to the analysis I will define software engineering as a field, as seen in the light of Wernick et al. and Moor \citep{Wernick:2004, Moor:1985}.

\subsection{Paradigms} \label{Theory:Paradigms}
When examining how real sciences evolve, Kuhn introduced the idea of a \textit{paradigm} \citep[p. 59]{Holm:2014}. Kuhn defines a paradigm as achievements that is
\begin{quote}
``sufficiently unprecedented to attract an enduring group of adherents away from competing modes of scientific activity ... [and] sufficiently open-ended to leave all sorts of problems for the ... practitioners to resolve'' \citep[p. 10-11]{Kuhn:2012}.
\end{quote}
A paradigm, then, is the new and unprecedented achievements, which opens up for novel understandings of scientific disciplines \citep[p. 61]{Holm:2014}. A set of ``rules, conventions, and common principles'' where a ``shared symbolic language'' and ``a number of scientific principles and basic laws'', makes up what Kuhn defines as a \textit{disciplinary matrix} (DM) [ibid.]. These rules are derived from the paradigm and exist to create methods of communication, so as to avoid situations like the raft sketch mentioned above \citep[p. 181]{Kuhn:2012}. As a result the ideas from two different paradigms becomes impossible to compare, since no facts can be shared directly between the two disciplinary matrices \citep[p. 66]{Holm:2014}. In other words they become incommensurable, where no one paradigm can be understood in the light of another, and where no paradigm can be said to be more `true' than the other, simply because there are no paradigm-independent (comparable) facts [ibid.].

Apart from the DM paradigms also consist of \textit{exemplars}, which constitutes the ``fundamental examples of the validity of the paradigm'' [ibid., p. 62]. These examples exists to train practitioners in the methods and DM of the paradigm, thus defining and enforcing the rules and conventions for that paradigm. Holm mentions x-rays as an example, where the untrained eye is hard pressed to find small abnormalities [ibid.]. Kuhn would explain this by doctors having years of training within a certain DM, and thus a certain `assimilation' a certain way of seeing \citep[p. 189]{Kuhn:2012}.

\subsection{Ethics}
Holm defines ethics and morals to include the ``unwritten rules that regulate our coexistence with other people'' [ibid., p.205]. Ethics can be seen as the implied rules we have been taught to judge what is \textit{right} and what is \textit{wrong}.

I have chosen utilitarianism as the main ethical theory because it is sufficiently simple and 'fair' in treating both sides of an argument. Further, it is able to cover a large part of the arguments I wish to bring into this essay. There are, however, some caveats, as I will discuss in section \ref{Discussion}.

\subsubsection{Utilitarianism}
To act rational as a autonomous actor can be defined as chosing the \textit{best} option out of several, that gives the actor most of what he or she wants \footnote{By this definition 'wanting' is subjective to the actor.} \citep[p. 232]{Gilje:2007}. Utilitarianism extends this model and states that rational actors will always seek to maximise the utility for the people involved, given the available knowledge \citep[p. 232]{Holm:2014, Gilje:2007}. Maximising happiness (as an example of a utility) can then be seen as an ethical 'guideline', where
\begin{quote}
``the action that is morally right is the one that results in the greatest possible utility'' \citep[p. 207]{Holm:2014}.
\end{quote}

%Copyright and software ownership can be considered as 

\section{Analysis}
Having outlined the theoretical framework, I will now apply the theories on the concepts I motivated in the introduction. 
To begin with I will attempt to understand why these difficulties in communication to and from IS related topics occurs. By including Wernick I will account for the position on SE as one or more paradigms, and the implication for the general understanding of SE for the surrounding world.
In the second part of the analysis I will use Moor to outline the concept of computer ethics and examine its consequences the general understanding of SE.

\subsection{SE as a paradigm}
Wernick et al. uses Kuhn's paradigms to examine the SE discipline and found evidence that supports a Kuhnian perspective, with an established DM as an important indicator \citep[p. 240]{Wernick:2004}. By analysing a number of textbooks and conducting interviews, Wernick et al. identified 20 beliefs present in both the theoretical and practical world [ibid.], supporting ``the existence of an underlying belief system for SE'' [ibid., p. 241].

The finding is interesting because it indicates that the existence of a common DM, establishes SE as a paradigm. Since paradigms and their inherent rules and exemplars are incommensurable, there can be no way of discussing or debating the concepts within the paradigm without using the DM and exemplars of that paradigm.

\subsection{SE and computer ethics}
Moor presents the idea of computer ethics (CE) in the subject of computer technology \citep{Moor:1985}. Since the development of computers and computer technology provides us with new capabilities, a policy and conceptual vacuum will appear, which, Moor argues, should be addressed by computer ethics [ibid.]. The viewpoint is interesting because it allows traditional theories on moral and ethics to be applied to computer technology, as if the technology is an ethical actor itself \citep{Floridi:1999}. In the words of Floridi:
\begin{quote}
``There has been a fundamental blind spot in our ethical discourse, which ... CE seems to be able to perceive and take into account [ibid., p. 55-56].
\end{quote}
While the field of CE is not as established as some of the older ethical traditions \footnote{For instance Floridi claims that CE can be understood as an object-oriented ontologically centered tradition, which is an interesting, but somewhat bold claim that I am not sure Moor would share \citep{Floridi:1999}.}, it might help us, as the quote above hints, to establish some ethical `guidelines' when working with automated (and autonomous) processes  \citep{Jensen:2014}.

\subsection{Case analysis: Debate on privacy}
In the raft sketch from the introduction the problem is, of course, that the two actors constantly talks past each other; they do not speak about the same things, or, in the terms of Kuhn, act within the same disciplinary matrix \citep{Holm:2014}. The raft sketch illustrates the consequences of claiming that SE can be seen as a paradigm: no one can engage in a discussion about the ideas of SE without understanding its DM.

In the debate where the Danish politician discussed surveillance with a Danish blogger (representing the technical community), the blogger starts by stating that 3.5 billion recordings of internet traffic has been made in the past few years. The blogger uses this figure to demand that the laws on logging are revised, but neither his number nor his claim are being commented by the politician. The blogger claims the politician does not understand the severity of the situation and resorts to calling her alienated and stupid. The politician sets the critique aside as a personal and irrelevant attack, and continues to motivate her own viewpoints.

Seeing SE as a paradigm explains the lack of communication between the politician and the blogger. The fact that the blogger uses terms and numbers from his own DM, means that it is impossible for the politician to understand and discuss, because she is not a part of the same paradigm.

In a previous clash with the technology community, the same politician argued that a service called Tor should be banished by law.

In another perspective the case can be understood in the light of ethics. If we assume that the hacker is arguing from the viewpoint of utilitarian information ethics, where open and public data is healthy and good, his argument makes sense; for him doing `good' means maximising the freedom for as many as possible. On the other hand the argument of the politician is that the police uses the data to capture criminals, which must be for the good of the society, and which justifies a minor reduction in freedom. 

There is an interesting common denominator in the \textit{rationality} of the CE perspective, which cannot be found with paradigms.

\section{Discussion} \label{Discussion}
Discuss caveats of utilitarianism.
Normal science.

Wernick: The analysis also shows that there the field of SE can be further divided into schools of beliefs, 

Development of Kuhn's paradigms. ``Kuhn went from seeing paradigms as dogmas ... to paradigms as exemplars'' \citep[p. 49]{Banville:1989}.

Hirscheim: "it was not possible to relate systems development methodologies to paradigms in this article" \citep[p. 1214]{Hirschheim:1989}.

\section{Conclusion}
In this paper I have shown that...
By analysing and discussing the question from the perspective
of philosophy of science I have ...

% TODO: Teori fra Moor og Wernick op i teoriafsnittet?
% TODO: SE or IS? What is the difference?

\bibliography{references}

\end{document}
