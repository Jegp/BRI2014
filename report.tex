\documentclass{article}

\usepackage{natbib}
\bibliographystyle{plainnat}
\usepackage{url}
\usepackage{nameref}

\begin{document}

\section{Introduction}
In an old Danish sketch about a sailing raft on actor, actor \texttt{A}, tries to explain a theatrical scene to another actor, actor \texttt{B}. The sketch begins by \texttt{A} explaining how a scene with a raft ends in a humorous climax. \texttt{B} stops \texttt{A} in his storytelling to asks what a 'climax' means. Actor \texttt{A} explains (slightly annoyed) that a climax is as an abstract peak of events. \texttt{B} nods and repeats that the raft sketch contains an abstract mountain. \texttt{B} then asks what \textit{abstract} means and \texttt{A} (even more annoyed) explains that \textit{abstract} is something that is not there. The confusion is complete when \texttt{B} surely concludes that the raft scene contains a mountain that is not there. Needless to say, the conversation does not benefit either \texttt{A} or \texttt{B} \citep{Raft:2014}.

This type of '\textit{raft-talk}' is not uncommon when experts in information systems (IS) encounters people outside the lingo. One example is a debate between a danish politician and a danish blogger about the recent surveillance scandals \citep{Bramsen:2014}. The interview has been heavily debated on social medias and led to some aggression because the IS felt misunderstood (ibid.).

In this essay I will attempt to analyse why these difficulties in communication to and from IS related topics occurs.
By using Wernick et al. and Thomas Kuhn's theory on paradigms \citep{Wernick:2004, Holm:2014}, I will examine if, and if is how, the field of software engineering (SE) and IS can be understood as incommensurable paradigms.

By applying utilitarianism and Moore's text on 'computer ethics' \citep{Moor:1985}, I will discuss the above ideas and test if the 'old' non-digital society have any common grounds with the 'new' digital.

Lastly I will discuss the two analyses and outline methods for further studies.

\section{Theory}
In this section I will include the theory about paradigms from Thomas Kuhn, and utilitarianism from Bentham and Mill. Before moving on to the analysis I will define software engineering as a field, as seen in the light of Wernick et al. and Moor \citep{Wernick:2004, Moor:1985}.

\subsection{Paradigms} \label{Theory:Paradigms}
When examining how real sciences evolve, Kuhn introduced the idea of a \textit{paradigm} \citep[p. 59]{Holm:2014}. Kuhn defines a paradigm as achievements that is
\begin{quote}
``sufficiently unprecedented to attract an enduring group of adherents away from competing modes of scientific activity ... [and] sufficiently open-ended to leave all sorts of problems for the ... practitioners to resolve'' \citep[p. 10-11]{Kuhn:2012}.
\end{quote}
A paradigm, then, is the new and unprecedented achievements, which opens up for novel understandings of scientific disciplines \citep[p. 61]{Holm:2014}. A set of ``rules, conventions, and common principles'' where a ``shared symbolic language'' and ``a number of scientific principles and basic laws'', makes up what Kuhn defines as a \textit{disciplinary matrix} (DM) [ibid.]. These rules are derived from the paradigm and exist to create methods of communication, so as to avoid situations like the raft sketch mentioned above \citep[p. 181]{Kuhn:2012}. As a result the ideas from two different paradigms becomes impossible to compare, since no facts can be shared directly between the two disciplinary matrices \citep[p. 66]{Holm:2014}. In other words they become incommensurable, where no one paradigm can be understood in the light of another, and where no paradigm can be said to be more `true' than the other, simply because there are no paradigm-independent (comparable) facts [ibid.].

\subsection{Ethics}
Holm defines ethics and morals to include the ``unwritten rules that regulate our coexistence with other people'' \citep[p. 205]{ibid.}. Ethics can be seen as the implied rules we have been taught to judge what is \textit{right} and what is \textit{wrong}.

I have chosen utilitarianism as the main ethical theory because it is sufficiently simple and 'fair' in treating both sides of an argument. Further, it is able to cover a large part of the arguments I wish to bring into this essay. There are, however, some caveats, as I will discuss in section \ref{Discussion}.

\subsubsection{Utilitarianism}
To act rational as a autonomous actor can be defined as chosing the \textit{best} option out of several, that gives the actor most of what he or she wants \footnote{By this definition 'wanting' is subjective to the actor.} \citep[p. 232]{Gilje:2007}. Utilitarianism extends this model and states that rational actors will always seek to maximise the utility for the people involved, given the available knowledge \citep[p. 232]{Holm:2014, Gilje:2007}. Maximising happiness (as an example of a utility) can then be seen as an ethical 'guideline', where
\begin{quote}
``the action that is morally right is the one that results in the greatest possible utility'' \citep[p. 207]{Holm:2014}.
\end{quote}

%Copyright and software ownership can be considered as 

\section{Analysis}
Having outlined the theoretical framework, I will now apply the theories on the concepts I motivated in the introduction. By including Wernick I will first account for the position on SE as one or more paradigms, and the implication for the general understanding of SE for the surrounding world.
In the second part of the analysis I will use Moor to outline the concept of computer ethics and examine its consequences the general understanding of SE.

\subsection{SE as a paradigm}
In the raft sketch from the introduction the problem is, of course, that the two actors do not speak about the same thing, or, in terms of Kuhn, act within the same disciplinary matrix \citep{Holm:2014}. Wernick et al. writes about 

\subsection{SE and computer ethics}
Moor 

\section{Discussion} \label{Discussion}
Discuss caveats of utilitarianism.

\section{Conclusion}
In this paper I have shown that...
By analysing and discussing the question from the perspective
of philosophy of science I have ...

\bibliography{references}

\end{document}
