\documentclass{article}
\begin{document}

\section{Introduction}
In its most fundamental form the question of ownership
of land, food and primary resources has existed for a
very very long time. Over the course of history,
moral and cultural rules have evolved to address the
many confusion and strives over this matter.
Faced with immaterialities such as
ideas and intellectual properties however, 'who owns what' becomes
an increasingly hard question to answer. Ideas can be thought of
by anyone, but the practical execution of them requires work
from the physical domain. A domain where \texttt{you} 

If intellect can be
owned by everyone, and intellect can reason about ideas, how can
ideas be owned? 

Copyright and software ownership can be considered as 

\section{Theory}
Property and ownership
Kuhn

\section{Analysis}

\section{Discussion}

\section{Conclusion}
In this paper I have shown that...
By analysing and discussing the question from the perspective
of philosophy of science I have ...

\end{document}
